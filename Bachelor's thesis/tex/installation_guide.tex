This is the guide on how to setup, run and deploy ElateMe back-end
server on Ubuntu OS.
{\parindent0pt
\section*{Requirements}\label{requirements}

To be able to setup and run project you need

\begin{itemize}
\tightlist
\item
  python3
\item
  pip
\item
  virtualenv
\end{itemize}

To install run:

\begin{verbatim}
sudo apt-get update
sudo apt-get install python3
sudo apt-get install python-pip
pip install virtualenv
\end{verbatim}

\pagebreak

\section*{Setup}\label{setup}

Clone repository:

\begin{verbatim}
git clone git@repo.micman.cz:allmywishes/server-api.git ElateMe
\end{verbatim}

Go to \texttt{ElateMe} folder and create virtual environment:

\begin{verbatim}
cd ElateMe
virtualenv -p /usr/bin/python3.5 venv
\end{verbatim}

To begin using the virtual environment, it needs to be activated:

\begin{verbatim}
source venv/bin/activate
\end{verbatim}

Install requirements inside virtual environment:

\begin{verbatim}
pip install -r requirements.txt
\end{verbatim}

Migrate Django models:

\begin{verbatim}
python manage.py makemigrations
python manage.py migrate
\end{verbatim}

By default project runs with \texttt{DEBUG=True} and SQLite database.

Run tests:

\begin{verbatim}
python manage.py test
\end{verbatim}

Now you should be able to run project locally:

\begin{verbatim}
python manage.py runserver
\end{verbatim}

Server should be running on \href{http://localhost:8000}{localhost:8000}
}
