%%%%%%%%%%%%%%%%%%%%%%%%%%%%% Foreword %%%%%%%%%%%%%%%%%%%%%%%%%%%%%
This chapter will focus on analysis of the project as a part of a software development that connects customers
requirements to the system and its following design and development.

Analysis of software project is intended to define detailed description of the product, break it down into
requirements to the system, their systematization, detection of dependencies, and documentation.

\newcounter{reqcounter}[section]
\newcommand{\req}[2]{
    \stepcounter{reqcounter}
    \indent\par
    \textbf{#1\arabic{reqcounter} #2}
}
\newcommand{\funcreq}[1]{\req{F}{#1}}
\newcommand{\nonfreq}[1]{\req{N}{#1}}

%%%%%%%%%%%%%%%%%%%%%%%%%%%%% Functional requirements %%%%%%%%%%%%%%%%%%%%%%%%%%%%%
\section{Functional requirements}

\subsection{Authorization}
\funcreq{Sign up via Facebook}
User will be able to sign up to ElateMe application with his Facebook account. Application will load users data
such as name, surname, email, date of birth, etc.
\funcreq{Logout}
Authorized user will be able to log out. In this case he will also stop receiving any notifications from the
application.
\funcreq{Load friends from social network}
On initial login application will load list of user's friends that are already signed up in this application. This
users will be considered as friends in ElateMe application.

\subsection{Friendship management}
\funcreq{View friends list}
User will be able to view list of his Facebook friends that are already signed up in application.
\funcreq{Create friends group}
User will be able to create friends group. Groups will be used for simplification of friends management.
\funcreq{Delete friends group}
User will be able to delete friends group.

\subsection{Wish management}
\funcreq{Create wish}
User will be able to create wish, set its title, description, price(amount of money that he(user) wants to gather),
and deadline.
\funcreq{Delete wish}
User will be able to delete his wish if nobody will have donated money yet.
\funcreq{Close wish}
User will be able to close his wish. Money that will have been gatherd on this wish will be refunded to donators.
\funcreq{View users' wishes list}
User will be able to browse wishes lists of his friends.
\funcreq{Create surprise wish}
User will be able to create surprise wish for one of his friends. In this case user to whom the wish was addressed
will not have acces to it and will not know about it until the whole amount is collected.
\funcreq{View contributed wishes list}
User will be able to view list of wishes he will have contributed to.

\subsection{Feed}
\funcreq{View users' feed}
User will recieve feed with latest wishes of his friends.

\subsection{Donation management}
\funcreq{Donate to wish}
User will be able to financially contribute to wishes of his friends.
\funcreq{Refund}
In the case of the closure of the wish, all gatherd money will be refunded to donators.

\subsection{Comments management}
\funcreq{View wishes comments list}
User will be able to view list of comments under the wish he will be browsing.
\funcreq{Comment wish}
User will be able to leave a comment under the wish.
\funcreq{Delete comment}
User will be able to delete his comment.

%%%%%%%%%%%%%%%%%%%%%%%%%%%%% Non-functional requirements %%%%%%%%%%%%%%%%%%%%%%%%%%%%%
\section{Non-functional requirement}

\subsection{Back-end \ac{API}}
\nonfreq{\ac{REST}ful}
Back-end API will follow architectural constraints of REST architectural style.
\nonfreq{\ac{HTTPS}}
Server will comunicate with client via \ac{HTTPS}.
\nonfreq{PostgreSQL database}
PostgreSQL will be used as the main DBMS.
\nonfreq{Performance}
Server will be able to serve 1500 requests per second.

\subsection{Payments}
\nonfreq{FIO-bank}
User will be able to make payments via FIO-bank.
\nonfreq{Bitcoin}
User will be able to make payments via Bitcoin.
\nonfreq{Secure payments}
System will ensure secure payments.
\nonfreq{Consistency}
Servers data about payments will be consistent with data in payments systems (FIO-bank, Bitcoin, etc.).
System will react accordingly to errors appeared during payments.

%%%%%%%%%%%%%%%%%%%%%%%%%%%%% Use cases %%%%%%%%%%%%%%%%%%%%%%%%%%%%%
\section{Use cases}
** insert Use cases diagram **



%%%%%%%%%%%%%%%%%%%%%%%%%%%%% System structure %%%%%%%%%%%%%%%%%%%%%%%%%%%%%
\section{System structure}
\image[1.3]{component_diagram.pdf}{Component diagram}

