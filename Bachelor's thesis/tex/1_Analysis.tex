%%%%%%%%%%%%%%%%%%%%%%%%%%%%% Foreword %%%%%%%%%%%%%%%%%%%%%%%%%%%%%
This chapter will focus on the analysis of the project as a part of a software development that connects customer's
requirements to the system and its subsequent design and development.

Analysis of software project is intended to define the detailed description of the product, break it down into
requirements to the system, their systema\-tization, detection of dependencies, and documentation.


%%%%%%%%%%%%%%%%%%%%%%%%%%%%% SP1 and SP2 subjects %%%%%%%%%%%%%%%%%%%%%%%%%%%%%
\section{BI-SP1 and BI-SP2 subjects}
The work on the ElateMe project started within the framework of the \textit{software team project} (BI-SP1 subject). Our
development team divided into groups:~Android, iOS, Backend developers, and architect. Our task was to~define and
document primary client's requirements, implement \m{functioning} prototypes of mobile applications and the server
\ac{API}. During BI-SP1~subject, Maksym Balatsko was working on the prototype of backend server, so the choice of used
technologies was up to him. Then the technology stack was agreed with the supervisor of the project. Chosen technologies
will be discussed in the next chapter.

Because of changes in requirements and the new interface design of the~mobile applications, analysis and its
documentation have undergone certain chan\-ges. And at the start of the BI-SP2 subject implementation of back-end
\ac{API} has begun.

\newcounter{reqcounter}[section]
\newcommand{\req}[2]{\setlength\itemsep{-0.1em}
\stepcounter{reqcounter}
\item[\textbf{#1\arabic{reqcounter}}] \textbf{#2}.}
\newcommand{\funcreq}[1]{\req{F}{#1}}
\newcommand{\nonfreq}[1]{\req{N}{#1}}

%%%%%%%%%%%%%%%%%%%%%%%%%%%%% Functional requirements %%%%%%%%%%%%%%%%%%%%%%%%%%%%%
\section{Functional requirements}
Functional requirements specify the behaviors the product will exhibit under specific conditions. They describe what
the developers must implement to enable users to accomplish their task (user requirements), thereby satisfying
the business requirements \cite{funcreq}.

\subsection*{Authorization}
\begin{itemize}
\funcreq{Sign up via Facebook}
User shall be able to sign up to the ElateMe application with his Facebook account. The application shall load user's
data such as name, surname, email, date of birth, etc.
\funcreq{Logout}
Authorized user shall be able to log out. In this case, he shall also stop receiving any notifications from
the application.
\funcreq{Load friends from social network}
On initial login application shall load a list of user's friends that are already signed up in this application. These
users shall be considered as friends in the ElateMe application.
\end{itemize}

\subsection*{Friendship management}

\begin{itemize}
\funcreq{View friends list}
User shall be able to view the list of his Facebook friends that are already signed up in the application.
\funcreq{Create friends group}
User shall be able to create friends group. Groups shall be used for simplification of friends management.
\funcreq{Delete friends group}
User shall be able to delete friends group.
\end{itemize}

\subsection*{Wish management}

\begin{itemize}
\funcreq{Create wish}
User shall be able to create a wish, set its title, des\-cription, price (amount of money that he (user) wants
to gather), and deadline.
\funcreq{Delete wish}
User shall be able to delete his wish if nobody has donated money yet.
\funcreq{Close wish}
User shall be able to close his wish. Money that will have been gathered on this wish shall be refunded to donators.
\funcreq{View users' wishes list}
User shall be able to browse wishes lists of his friends.
\funcreq{Create surprise wish}
User shall be able to create surprise wish for one of his friends. In this case, the user, to whom the wish was
addressed, shall not have access to it and shall not know about it until the whole amount is collected.
\funcreq{View contributed wishes list}
User shall be able to view the list of wishes he will have contributed.
\end{itemize}

\subsection*{Feed and notifications}
\begin{itemize}
\funcreq{View user's feed}
User shall receive the feed with the latest news of his friends.
\funcreq{View user's notifications}
User shall receive information about the state of his wishes, new donations, comments, etc. via push notifications.
\end{itemize}

\subsection*{Donation management}
\begin{itemize}
\funcreq{Donate to wish}
User shall be able to contribute to wishes of his friends financially.
\funcreq{Refund}
In the case of the closure of the wish, all gathered money shall be refunded to donators.
\end{itemize}

\subsection*{Comments management}
\begin{itemize}
\funcreq{View wishes comments list}
User shall be able to view the list of comments under the wish he will be browsing.
\funcreq{Comment wish}
User shall be able to leave a comment under the wish.
\funcreq{Delete comment}
User shall be able to delete his comment.
\end{itemize}

%%%%%%%%%%%%%%%%%%%%%%%%%%%%% Non-functional requirements %%%%%%%%%%%%%%%%%%%%%%%%%%%%%
\section{Non-functional requirement}
Non-functional requirement is a software requirement that describes not what the software will do, but how the software
will do it, for example, software performance requirements, software external interface requirements, software design
constraints and software quality attributes. Non-functional requirements are difficult to test; therefore they are
usually evaluated subjectively~\cite{nonfuncreq}.

\subsection*{Back-end \ac{API}}
\begin{itemize}
\nonfreq{\acs{REST}ful}
Back-end API shall follow architectural constraints of \acs{REST} architectural style.
\nonfreq{\acs{HTTPS}}
The server shall communicate with the client via \ac{HTTPS}.
\nonfreq{PostgreSQL database}
PostgreSQL shall be used as the primary \ac{DBMS}.
\nonfreq{Performance}
The server shall be able to serve 1500 requests per second.
\end{itemize}

\subsection*{Payments}
\begin{itemize}
\nonfreq{FIO-bank}
User shall be able to make payments via FIO-bank.
\nonfreq{Bitcoin}
User shall be able to make payments via Bitcoin.
\nonfreq{Secure payments}
The system shall ensure secure payments.
\nonfreq{Consistency}
Servers data about payments shall be consistent with data in payments systems (FIO-bank, Bitcoin, etc.). The system
shall react accordingly to errors appeared during payments.
\end{itemize}



\newcommand{\uccomponent}[1]{
\item \textbf{#1}
}
\newcommand{\ucactor}[1]{
\item \textbf{#1}
}
\newcommand{\ucgroup}[1]{
\item \textbf{#1}
}
%%%%%%%%%%%%%%%%%%%%%%%%%%%%% Use cases %%%%%%%%%%%%%%%%%%%%%%%%%%%%%
\section{Use cases}
Use cases were defined after analyzing of functional and non-functional requirements. Use cases documentation serves for
better understandings of functionality required from the system. Use cases model of this application is presented on
the diagram \ref{fig:use_cases_diagram}.

\image[1.3]{use_cases_diagram}{pdf}{Use cases diagram}

Use case model includes the following components:

\begin{itemize}
\uccomponent{Actors} represent people and external systems involved in a particular use cases. The diagram focuses
specifically on the actors to insure that the system provides useful and usable functionality.

The main actors of this system are:
\begin{itemize}
\ucactor{User} that uses mobile or web application.
\ucactor{Facebook} participates in user authorization and provides an inter\-face for obtaining the necessary
information about him (user).
\ucactor{Payment system} participates in payments, namely donations and redundancies. In the role of payment systems in
this application are FIO-Banka and Bitcoin.
\end{itemize}

\uccomponent{Use cases} represent the functionality that the system provides for actors. In this diagram, they are
divided into logical groups namely
\begin{itemize}
\ucgroup{Authentication} of the user through Facebook and obtaining the necessary information about the user from his
Facebook account.
\ucgroup{Friendship} between users and the division of friends into groups.
\ucgroup{Wish management} includes the creation of wishes, donations, and comments, as well as closure of the wishes
with the subsequent refund.
\ucgroup{Other} includes the news feed and notifications.
\end{itemize}

\end{itemize}


%%%%%%%%%%%%%%%%%%%%%%%%%%%%% Domain model %%%%%%%%%%%%%%%%%%%%%%%%%%%%%
\section{Domain model}
For simplification of understanding of the primary domain classes and their behavior, it was decided to define so-called
Domain model. The domain model is the visual representation of conceptual classes in a domain of interests. Domain model
is visualization of things in real-world, not of software components such as C++ or Python class \cite{domainmodel}.

Domain model \ref{fig:domain_model} was created by Maksym Balatsko, team lead and advert server developer of ElateMe.
In his bachelor's thesis \cite{adserver} he describes in detail this diagram, classes specified in it, their attributes,
and associations between them.

\image[1.1]{domain_model}{pdf}{Domain model \cite{adserver}}



%%%%%%%%%%%%%%%%%%%%%%%%%%%%% System structure %%%%%%%%%%%%%%%%%%%%%%%%%%%%%
\section{System structure}
The whole ElateMe application system is divided into components. Principal components are the server, Android, and iOS
clients.

The detailed structure of the server and its connection with external inter\-faces are presented at the component diagram
\ref{fig:component_diagram}. As seen in the diagram, the server provides interface for the mobile applications to
communicate via \ac{REST} \ac{API}. The server also uses interfaces of Facebook (Graph API) to receive needed data about
users and interfaces of payment systems (FIO-bank and Bitcoin) for payments processing.

Inside, the server is divided into components that are responsible for sto\-ring and processing data of application
entities. This components are called \textit{apps} in Django. Apps communicate with the database via Django
\textit{models}. Models in Django is an interface designed to simplify querying to the database.

\image[1.3]{component_diagram}{pdf}{Component diagram}

The diagram also shows the use of interfaces of Facebook and payment systems by mobile clients, but they are not a part
of my work, so their design and implementation will not be described in this thesis.



%%%%%%%%%%%%%%%%%%%%%%%%%%%%% Authentication %%%%%%%%%%%%%%%%%%%%%%%%%%%%%
\section{Authentication}
The user has to be authorized to use the application. The ElateMe application will not provide in-app registration. User
authentication will be conducted exclusively through third-party systems. It is made to simplify the registration in
the application.


\subsection{Facebook}
User authentication will be conducted through his Facebook account.

After the first login, the application will get from Facebook needed infor\-mation about the user: first name, last
name, email address, a list of user's friends. User's Facebook friends, who are already logged in to the application,
automatically become his friends in the ElateMe.

Despite the lack of in-app registration, user's information received from Facebook will be stored in ElateMe system as
well, because a user will be able to add other users to his friend list, create friends groups independently from
Facebook.



\newcommand{\uitem}[1]{
\item \textbf{#1} \par
}
\newcommand{\aitem}[1]{
\item \textbf{#1} \par
}
%%%%%%%%%%%%%%%%%%%%%%%%%%%%% Payments system %%%%%%%%%%%%%%%%%%%%%%%%%%%%%
\section{Payments system}
The ElateMe project is based on crowdfunding. So application (and the server in particular) has to provide service for
payments processing. According to the requirements, this service should use interfaces of FIO-banka and Bitcoin.

\subsection{Use cases}
In this application, the payment system participates in the following use cases:

\begin{itemize}
\uitem{Donation}
During the donation, the money is transferred to the internal account of ElateMe, where it is stored until one of
the~following use cases.

\uitem{Wish completion}
In the case of fulfillment of the wish, money is transferred to the account of the author of this wish.

\uitem{Wish closing}
There are two situations in which the wish is closed: closing of the wish by its author and closing upon expiration of
the deadline. In both cases, already collected money is returned to the donators' accounts.
\end{itemize}

\subsection{FIO-banka}
Since the open \ac{API} of the FIO-banka \cite{fio} does not provide sufficient functionality for the needs of this
application, it was necessary to agree directly with the bank on the provision of required interface. Due to prolonged
communication with the bank, we still did not get the full documentation. We only got a description of the payment
process, which can be seen on the diagram \ref{fig:fio_payment_activity}.

\image[1.2]{fio_payment_activity}{pdf}{Payment via FIO-banka}

As seen in the diagram, before initiating the payment, the mobile application will send a request to the server to
create a donation; the server will create an unconfirmed donation and respond to the application with information about
the~newly created donation. From this information, the~application will take the donation id, which will later serve as
the~identifier of the paid product. The application initializes the payment with \m{the necessary} \m{information} about
the user and donation and redirects the user to the payment gateway of the~FIO-banka in the web-view. After processing
the payment, the~FIO-Banka redirects the user to one of the two URLs on the server, depending on the~state of
the payment (success/fail URLs). In the case of~success, the server marks the payment as confirmed, alternatively
failed. The mobile application will close the web-view and notify the user about the status of the payment.

\subsection{Bitcoin}
The second way to make payments in the application is Bitcoin.

Bitcoin is a collection of concepts and technologies that form the basis of a digital money ecosystem. Units of currency
called bitcoins are used to~store and transmit value among participants in the bitcoin network. \m{Bitcoin} users
communicate with each other using the bitcoin protocol primarily via the~Inter\-net, although other transport networks
can also be used. \m{The bitcoin} protocol stack, available as open source software, can be run on wide range of
computing devices, including laptops and smartphones, making the technology easily accessible \cite{bitcoin}. A detailed
analysis of the system of Bitcoins was carried out in the bachelor's thesis of Yegor Terokhin \cite{ios1}, one of
the iOS developers of ElateMe. In his work, he describes in detail the principles of work of Bitcoin and the benefits
of using this system in the framework of the ElateMe project.

Yegor decided to use Coinbase, to process Bitcoin payments. Founded in June of 2012, Coinbase is a digital currency
wallet and platform where merchants and consumers can transact with new digital currencies like bitcoin and ethereum
\cite{coinbase}. The main benefit of using Coinbase is that it provides \ac{SDK} for both mobile platforms and a Python
used for the implementation of the backend in this project.

\subsection{Refund mechanism}
As was mentioned before, after closing of the wish, collected money will be refunded to the donators' accounts. ElateMe
will have a bank account or online wallet for every payment system that it will provide. As every payment system has
a different interface for conducting payments, every donation and its refund will use a different way of payment
processing, depending on the available interface for each particular system. There are three possible solutions on
processing refunds:

\begin{itemize}\setlength\itemsep{.2em}
    \item \ac{SDK} (e.g. Coinbase \ac{SDK})
    \item \acs{HTTP} \ac{API} (e.g. FIO-banka \ac{API})
    \item Web user interface (e.g. mBank)
\end{itemize}

As long as payment system provides \ac{SDK} or any other \ac{API}, the implementation of donation and refunds processing
does not require complicated systems to conduct payments. On the current stage of development, it was decided to use
only FIO-banka and Bitcoin, which provide appropriate interfaces. But in the case of the addition of payment systems
that don't have \ac{API} for developers, payments, and refunds, in particular, need to be processed through other
available interfaces.

Following solution was proposed by Michal Maněna, project supervisor of ElateMe.

User web interfaces of payment systems that don't provide proper \ac{API} will be used for conducting refunds. It was
decided to use Selenium for automatization of payments via user web interface. \definition{Selenium} is a set of tools
specifically for automating web browsers \cite{selenium}. So it will simulate actions directly in the web browser
to provide refunds to the users' bank accounts. Here it is necessary to reckon with the fact that many banks send
a confirmation code via SMS. In this case, it is required to have a \ac{GSM} module with a SIM card physically connected
to the server. Then there will be running daemon that will collect and save the~SMS messages in the database. Then
confirmation code will be entered in the~corresponding field by Selenium and sent to process payment.


\section{Push notifications}
\definition{Push notifications} are short important messages from the application or service, displayed by the operating
system when the user does not directly work with the specified application or service. The advantage of such
notifications is that there is no need to keep the program in memory, spending on it processor powers and memory.

In the ElateMe application, the user will receive information about the state of his wishes, new donations, comments,
etc.

\subsection{Mechanism of push notifications}
For the server to be able to send push notifications it needs to store a token. \definition{Token} is a line of
characters that serves as an address of specific application on the particular device. The token is generated by
\ac{OSPNS}. After the application is installed on the device it registers itself for receiving of push notifications;
OS requests token from \ac{OSPNS}, the application receives token and sends it to the server.

\image[1]{push_notifications_mechanism}{pdf}{Mechanism of push notifications}

\subsection{Actors}

\begin{itemize}
\aitem{\ac{OSPNS}}
Every operating system has its service for processing push notifications. They are \ac{GCM} for Android and \ac{APNs}
for iOS. As shown on the diagram \ref{fig:push_notifications_mechanism} OSPNS sends a token to the application when it
registers in the service and sends the push notifications to the application itself.

\aitem{Server}
The server stores the tokens of each device and sends the push notifications to OSPNS.

\aitem{Client application}
The application is registered to receive a push notification, receives a~token from OSPNS and sends it to the server.
\end{itemize}






