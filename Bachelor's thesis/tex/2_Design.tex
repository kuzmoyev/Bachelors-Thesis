%%%%%%%%%%%%%%%%%%%%%%%%%%%%% Authentication %%%%%%%%%%%%%%%%%%%%%%%%%%%%%
\section{Authentication}
\subsection{OAuth 2.0}
OAuth 2.0 is the industry-standard protocol for authorization. OAuth 2.0 supersedes the work done on the original OAuth
protocol created in 2006. OAuth 2.0 focuses on client developer simplicity while providing specific authorization flows
for web applications, desktop applications, mobile phones, and living room devices. ** https://oauth.net/2/ **

For server to be able to get list of friends and other information about user, mobile application needs to recieve
token from facebook with appropriate prmissions and send it to the server. Token is a line generated by facebook and
by which facebook provides access to certain data of certain user.

%%%%%%%%%%%%%%%%%%%%%%%%%%%%% Server API %%%%%%%%%%%%%%%%%%%%%%%%%%%%%
\section{Server API}

\subsection{REST}
\ac{REST} is the architectural solution for the transfer of structural data between server and client.
\subsection{URL scheme}
\subsection{Apiary}


%%%%%%%%%%%%%%%%%%%%%%%%%%%%% Chosen technologies %%%%%%%%%%%%%%%%%%%%%%%%%%%%%
\section{Chosen technologies}
As I mentioned before, the choice of used technology was not up to me so in this section I will not describe why
certain technologies were chosen, but will describe their advantages (alternatively disadvantages) for this project.


\subsection{Python}
Python is a base of the server. It was chosen as a primary programming language because it was designed to be simple
and highly readable which is very important for large-scale projects. Its syntax and standard library simplify and
speed up a development.

\subsection{Django}
Django is an open source web framework for python. It provides a high level abstraction of common web development
patterns. It follows \ac{MVC} design pattern. Django uses \ac{MVC} to separate model as a data and a business logic of
the application, view as a representation of the information for the user, in this case, the client side of the
application and controller as an interface of the application, in this case, set of URLs to communicate with front-end.

\subsection{Django REST}
Django REST framework is an open source tool built on Django framework. It contains needed tools for implementation of
\ac{REST}ful \ac{API} and follows all constraints of \ac{REST}ful server mentioned earlier.


\subsection{PostgreSQL}
On initial stage of the development, SQLite will be used as a \ac{DBMS}, because it does not require a standalone
database server and is simple to set up. The database will be changed and migrated to PostgreSQL later.\par
PostgreSQL\cite{postgres} is powerful, open source relational \ac{DBMS}. It has advanced features such as full
atomicity, consistency, isolation, durability. Django framework provides great \ac{API} for working with PostgreSQL
databases.


