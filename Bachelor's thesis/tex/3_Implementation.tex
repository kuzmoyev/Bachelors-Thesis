This chapter contains a~description of the~implementation of the~project's server side. This part will describe
the~structure of the~project. It is intended to familiarize the~reader with the~implementation of this application and
to simplify the~understanding of the~structure of the~project for future developers.

Note that the~installation guide for this application is in the~attachment~\ref{ch:installation_guide}.


\newcommand{\appitem}[1]{
\item\textbf{#1}.}
\newcommand{\setitem}[1]{
\item\textbf{#1}.}
%%%%%%%%%%%%%%%%%%%%%%%%%%%%% Django project structure %%%%%%%%%%%%%%%%%%%%%%%%%%%%%
\section{Project structure}
Django as a~framework determines the~structure of the~whole system. Django project is divided into logical parts,
\textit{apps}. \definition{Apps} contain a~set of modules with classes, which implement interfaces and extend classes,
which are provided by Django.

Later in this section, the~main modules of the~apps in the~Django project will be described. The~parts involved in
the~processing of the~request for the~receiving of the~user's wishes will be taken as an example.

\subsection{Django models}
Django \definition{models} is an interface for simplified querying to the~database. All models extend class
\class{model} from \module{django.db} module and usually represent single table in a~database \cite{djangodocs}.

Thus, each app, except the~\app{feed} and \app{notifications}, contains a~module \module{models}. In this module, there
are models that completely describe the~database. Despite what kind of \ac{DBMS} is used (PostgreSQL or SQLite)
the~Django models and querying through them does not change, which simplifies development, testing and deploy.
\pagebreak

Model that represent table \class{Wish} in the~database:
\begin{lstlisting}
class Wish(models.Model):
    title = models.CharField(max_length=100)
    description = models.CharField(max_length=512)
    amount_needed = models.FloatField()
    date_created = models.DateTimeField(auto_now_add=True)
    date_of_expiration = models.DateTimeField(null=True)
    date_completed = models.DateTimeField(null=True)
    is_public = models.BooleanField(default=False)
    author = models.ForeignKey(User,
                               on_delete=models.CASCADE,
                               related_name='wishes')

    class Meta:
        db_table = 'Wish'

\end{lstlisting}

\subsection{Django views}
Django \definition{view} is a~method that is called during request on certain \ac{URL}. This function takes a~Web
request and returns a~Web response, in this case, \ac{JSON}. The~main logic of processing requests is in the~views.

Before we started using Django REST framework, the~request on receiving the~wish list of the~current user looked like this:
\begin{lstlisting}
def current_user_wishes_view(request):
    current_user = request.user
    if not user.is_authenticated():
        return HttpResponse('Unauthorized', status=401)

    current_user_wishes = current.user.wishes. \
                                     order_by('-date_created')

    response_data = []
    for wish in current_user_wishes:
        serialized_wish = WishSerializer(wish).data
        response_data.append(serialized_wish)

    paginated_response = WishPagination(). \
               get_paginated_response(request, response_data)

    return JsonResponse(paginated_response)

\end{lstlisting}
Such requests, to obtain a~list of objects of a~certain model (wishes, donations, comments, etc.), look very similar. It
is checked if the~user is authenticated, the~data queryset is obtained, the~data is serialized, paginated (divided into
pages), returned in the~\ac{JSON} format. It was decided to use the~Django REST framework, to simplify
the~implementation of such requests and the~corresponding auxiliary classes (serializers, paginations, etc.)

Thus, in Django REST framework view on getting the~current user's wish list looks like this:

\begin{lstlisting}
class CurrentUserWishesView(generics.ListCreateAPIView):
    renderer_classes = (renderers.JSONRenderer,)
    permission_classes = (permissions.IsAuthenticated,)
    serializer_class = serializers.WishSerializer
    pagination_class = pagination.WishPagination

    def get_queryset(self):
        user = self.request.user
        return user.wishes.order_by('-date_created')
\end{lstlisting}

This approach simplifies implementation and improves the~readability of the~code.


\subsection{Django urls}
The \definition{urls} module in Django is responsible for linking the~\ac{URL} endpoints to their corresponding views.
It contains a~list of objects \class{url}. In \app{wishes} app it looks like this:

\begin{lstlisting}
urlpatterns = [
    url(r'wishes/', CurrentUserWishesView.as_view()),
    # other urls
]
\end{lstlisting}


\subsection{Django REST serializers}
Django REST \definition{serializers} is an interface that provides the~Django REST framework for simplifying
the~serialization and deserialization of instances of Django models. The~simplest wish serializer looks like this:

\begin{lstlisting}
class WishSerializer(serializers.ModelSerializer):
    class Meta:
        model = Wish
        fields = '__all__'
        read_only_fields = ('id', 'author',
                            'date_created', 'date_completed',
                            'amount_gathered', 'donators_count')
\end{lstlisting}
\pagebreak

\subsection{Django REST pagination}
It was decided to use pagination, to avoid large responses in the~case of a~big number of objects in the~queryset.
\definition{Pagination} is the~partitioning of the~response into so-called pages of the~same size. It is enough to
extend the~\class{PageNumberPagination} class from the~module \module{rest\_framework.pagination} to create
a~class respon\-sible for the~pagination of the~list of data:

\begin{lstlisting}
class WishPagination(PageNumberPagination):
    page_size = 10
    page_size_query_param = 'page_size'
    max_page_size = 50
\end{lstlisting}

If this class is used as \field{pagination\_class} in the~view, \field{page\_size} and \field{page} are used in
the~\ac{URL} as optional parameters.

So on the~request ``\textit{/wishes?page\_size=5\&page=2}'' server will respond with \ac{JSON} in the~following format:

\begin{lstlisting}
{
  "count": 13,
  "next": "https://api.elateme.com/wishes?page_size=5&page=3",
  "previous": "https://api.elateme.com/wishes?page_size=5&page=1",
  "results": [
    # 5 serialized wishes from the second page
  ]
}
\end{lstlisting}


\subsection{Django apps}
The project was divided into the~following apps:

\begin{itemize}

\appitem{account} App includes modules for storing and processing information about the~user. \app{account} is divided
into sub-applications \app{authorization} and \app{social} that are responsible for user authorization and integration
with social networks respectively. \appitem{donations} This app is designed to process donations. It will also contain
the~logic of the~payment and refund systems.
\appitem{feed} App for the~arrangement of a~user's news feed.
\appitem{friendship} Application for the~processing of friendly relationships between users.
\appitem{notifications} App provides user notifications. At the~moment, it provides the~\ac{REST} interface for getting
news list. Later this appli\-cation will work with push notifications.
\appitem{wishes} Application provides the~interface for processing user wishes. It also contains sub-application
\app{comments}.

\end{itemize}

\subsection{Django settings}
Django \definition{settings} is a~module that contains all the~configuration of the~Django project. The~main
configurations to notice are:

\begin{itemize}\setlength\itemsep{-0.2em}
    \setitem{INSTALLED\_APPS} A~list of all apps in a~project.
    \setitem{ALLOWED\_HOSTS} A~list of the~host/domain names that this Django site can serve.
    \setitem{DATABASES} A~dictionary containing the~settings for all databases to be used with Django.
    \setitem{DEBUG} A~boolean that turns on/off debug mode.
\end{itemize}

A complete list of settings available in Django can be found in the~official documentation \cite{djangosettings}.

As for development and deployment it is needed to have different confi\-gurations for allowed hosts, databases and
debugging, in the~\file{server\_api} folder, alongside with \module{settings} module there were created two modules:
\module{prod\_settings} and \module{dev\_settings}. They contain specific configurations for the~production and
development respectively. For example \module{dev\_settings} have configured \m{SQLite} database and set debug mode on,
while \module{prod\_settings} defines settings for \m{PostgreSQL} without debug mode. To enable \module{prod\_settings}
it is needed to set environment variable \textit{PRODUCTION}. Otherwise \module{dev\_settings} will be used.

Following lines were added to the~main \module{settings} module to make it work:

\begin{lstlisting}
if os.environ.get('PRODUCTION', False):
    from .prod_settings import *
else:
    from .dev_settings import *
\end{lstlisting}



%%%%%%%%%%%%%%%%%%%%%%%%%%%%% Python Virtual Environment %%%%%%%%%%%%%%%%%%%%%%%%%%%%%
\section{Python Virtual Environment}
Since this application uses a~set of dependencies that don’t come as a part of the~Python standard library they must be
installed for all instances of the~application, namely the~development and testing on the~local machines of developers
and the~production server. Python Virtual Environment was used for these purposes.

A \definition{Virtual Environment} is a~tool to keep the~dependencies required by different projects in separate places,
by creating virtual Python environments for them. It solves the~“\textit{Project X depends on version 1.x but, Project Y
needs 4.x}” dilemma, and keeps global site-packages directory clean and manageable~\cite{pythonvenv}.

A list of all the~dependencies that are installed in the~virtual environment can be found in the~\file{requirements.txt}
file in the~project's root directory. The~guide for installing and configuring the~virtual environment is described in
the~installation guide in the~attachment \ref{ch:installation_guide}.
